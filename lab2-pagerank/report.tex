\documentclass[a4paper]{article}
\usepackage{ctex}
\usepackage{enumitem}
\usepackage{multirow}
\usepackage{fancyhdr}
\usepackage{amsmath}
\usepackage{parskip}
\usepackage{float}

\setlength{\parskip}{6pt}

\pagestyle{headings}

\begin{document}
\title{实验二:PageRank算法实现}
\author{梁业升 2019010547(计03)}

\maketitle

\section{实验结果}

\begin{figure}[!htbp]
    \centering
    \includegraphics[width = \textwidth]{assets/oj.png}
    \caption{OJ测试结果}
\end{figure}

\section{实验报告}

\subsection{请分析在迭代过程中,为什么PageRank值的和始终为1}

第 $k(k\ge 1)$ 次迭代中,出度为 0 的节点的 PageRank 值之和为

\begin{equation*}
    S^{(k)}=\sum_{\{j|j\Rightarrow i\}=\emptyset}\texttt{PR}[i]
\end{equation*}

节点 $i$ 的 PageRank 值为

\begin{equation*}
    \texttt{PR}[i]^{(k)}=\frac{\alpha}{N}+(1-\alpha)
        \Big [\sum_{j\Rightarrow i}\frac{\texttt{PR}[j]^{(k-1)}}{\texttt{out\_degree}[j]}
     + \frac{S^{(k)}}{N} \Big]
\end{equation*}

PageRank的和为

\begin{equation*}
    \begin{split}
        \sum_{i\in V(G)} \texttt{PR}[i]^{(k)} & = N\cdot\frac{\alpha}{N}
            + (1-\alpha) \Big[\sum_{(j,i)\in E(G)}\frac{\texttt{PR}[j]^{(k-1)}}{\texttt{out\_degree}[j]}
            + \sum_{\{i|j\Rightarrow i\}=\emptyset}\texttt{PR}[j]^{(k-1)} \Big]
        \\ & = \alpha + (1-\alpha)\sum_{\{i|j\Rightarrow i\}\neq\emptyset}\texttt{PR}[j]^{(k-1)}
            + (1-\alpha)\sum_{\{i|j\Rightarrow i\}=\emptyset}\texttt{PR}[j]^{(k-1)}
        \\ & = \alpha + (1-\alpha) \sum_{i\in V(G)} \texttt{PR}[i]^{(k-1)}
    \end{split}
\end{equation*}

$k = 0$ 时,

\begin{equation*}
    \sum_{i\in V(G)} \texttt{PR}[i]^{(0)} = 1
\end{equation*}

由数学归纳法

\begin{equation*}
    \sum_{i\in V(G)} \texttt{PR}[i]^{(k)} = \alpha + (1-\alpha) = 1
\end{equation*}

\subsection{语料入链接数和出链接数分布情况分布}

\begin{figure}[H]
    \centering
    \includegraphics[width = \textwidth]{plots/in.png}
    \caption{入链接数分布}
\end{figure}


\begin{figure}[H]
    \centering
    \includegraphics[width = \textwidth]{plots/out.png}
    \caption{出链接数分布}
\end{figure}    

大多数的网页入链接和出链接数均较少,具有较高入链接或出链接的网页较少。相较而言,入链接数较少的网页
的比例比出链接数较少的网页的比例多,而最大出链接数量少于最大入链接数量。

\subsection{PageRank结果分布}

\begin{figure}[H]
    \centering
    \includegraphics[width = \textwidth]{plots/pagerank.png}
    \caption{PageRank结果分布}
\end{figure}    

大多数的PageRank均较低,具有较大PageRank的网页较少。

\subsection{PageRank得分与入链接的关联分析}

\begin{figure}[H]
    \centering
    \includegraphics[width = \textwidth]{plots/in-pr.png}
    \caption{PageRank得分与入链接的关系}
\end{figure}    

由上图可见,PageRank得分与入链接数量大致成正相关,且入链接越多,相关性越强。

\subsection{PageRank得分与相应条目语义内容分析}

PageRank 排名前 3 的条目为:

\begin{enumerate}
    \item 箭头:$1.87\times 10^{-2}$
    \item $\leftarrow$:$1.59\times 10^{-2}$
    \item 维基数据:$3.1\times 10^{-3}$
\end{enumerate}

PageRank 排名后 2 的条目为:

\begin{enumerate}
    \item 李东濬:$1.586\times 10^{-7}$
    \item 金炯:$1.586\times 10^{-7}$
\end{enumerate}

可见,PageRank较高的条目为搜索或引用的高频词,较低的条目大多为如部分姓名的冷门词。

\end{document}
